\documentclass{article}


\usepackage{amsmath}
\usepackage{amssymb}
\usepackage{mathtools}
\usepackage{mathrsfs}

\setlength{\parindent}{0pt}

\newcommand{\dep}[1]{{\scriptstyle(#1)}}

\begin{document}


\section{Osnovne enačbe}
$$\mathbf{u} = 
\begin{bmatrix}
	u_x\\
	u_z\\
	\varphi
\end{bmatrix} \qquad \mathbf{v} = 
\begin{bmatrix}
	v_x\\
	v_z\\
	\Omega
\end{bmatrix} \qquad \mathbf{e}_0=
\begin{bmatrix}
	-1\\
	0\\
	-\kappa_0
\end{bmatrix} \qquad \mathbf{d} = 
\begin{bmatrix}
	u_x' + \cos\varphi_0\\
	u_z' + \sin\varphi_0\\
	\varphi'
\end{bmatrix} \qquad \mathbf{p} = 
\begin{bmatrix}
	p_x\\
	p_z\\
	m_y
\end{bmatrix}
$$


$$\mathbf{R}\dep{\varphi} =
\begin{bmatrix}
	\cos\varphi & \sin \varphi & 0\\
	-\sin\varphi & \cos\varphi & 0\\
	0 & 0 & 1
\end{bmatrix} \qquad \mathbf{R}^{(n)}\dep{\varphi} = 
\begin{bmatrix}
	0 & 1 & 0\\
	-1 & 0 & 0\\
	0 & 0 & 0
\end{bmatrix}^n \mathbf{R}\dep{\varphi}
$$


$$\mathbf{e} = \mathbf{R}\dep{\varphi}\mathbf{d}+\mathbf{e}_0 \qquad \mathcal{R} = \mathbf{R}^T\dep{\varphi}\mathbf{Ce} \qquad \mathcal{N} = \mathbf{Ce}$$


$$\delta \mathbf{e} = \delta\varphi\;\mathbf{R}^{(1)}\dep{\varphi}\mathbf{d} + \mathbf{R}\dep{\varphi}\;\delta \mathbf{d}$$
$$\delta \mathcal{R} = \delta\varphi\;((\mathbf{R}^{(1)})^T\dep{\varphi}\mathbf{Ce}+\mathbf{R}^T\dep{\varphi}\mathbf{CR}^{(1)}\dep{\varphi}\mathbf{d}) +  \mathbf{R}^T\dep{\varphi}\mathbf{CR}\dep{\varphi}\;\delta \mathbf{d}$$
$$\delta \mathcal{N} = \delta\varphi\;\mathbf{CR}^{(1)}\dep{\varphi}\mathbf{d} +  \mathbf{CR}\dep{\varphi}\;\delta \mathbf{d}$$


$$\mathcal{F} = \int_0^l-\mathcal{R}\mathcal{P}_i' + (\mathbf{p}-\rho_A \dot{\mathbf{v}} + {\scriptscriptstyle \begin{bmatrix} 0\\ 0\\ \mathcal{N}_1\mathbf{e}_2 - (1+\mathbf{e}_1) \mathcal{N}_2 \end{bmatrix}}) \mathcal{P}_i \; dx + \mathcal{R}\mathcal{P}_i\Bigl |_0^l$$
	$$\delta \mathcal{F} = \int_0^l- \delta \mathcal{RP}_i' + {\scriptscriptstyle \begin{bmatrix} 0\\ 0\\ \mathbf{e}_2\delta\mathcal{N}_1+\delta \mathbf{e}_2 \mathcal{N}_1-(1+\mathbf{e}_1)\delta \mathcal{N}_2 - \delta \mathbf{e}_1 \mathcal{N}_2 \end{bmatrix}}\mathcal{P}_i\; dx$$

Če vzamem $\mathbf{e} = [C,0,0]$, kejr je $C>0$ dobim po formuli $\mathbf{d} = [0,C+1,0]$.

Recimo, da je element vertikalen, in spodaj podprt. Potem je $u_z$ linearna funkcija, ki pada z koordinato $x$ na elementu. Tako je $\mathbf{d} = [0,A+1,0]$, kjer je $A<0$ kar je v protislovju.

Rezultata se uskaldita za $\mathbf{e} = \mathbf{R}^T\mathbf{d}+\mathbf{e}_0$.





\newpage
\section{Numerični račun}
Poznamo količine v času $t_n$. Za $t_{n+1}$ predpostavimo hitrosti $\mathbf{v}_{n+1}$.

$$\bar{\mathbf{v}} = (\mathbf{v}_{n+1} + \mathbf{v}_{n})/2 \qquad \bar{\mathbf{u}} = \mathbf{u}_{n} + \frac{h}{2}\bar{\mathbf{v}} \qquad \bar{\mathbf{d}} = \mathbf{d}_n + \frac{h}{2}\bar{\mathbf{v}}' $$
$$\mathbf{u}_{n+1} = \mathbf{u}_n+h\bar{\mathbf{v}} \qquad \mathbf{d}_{n+1} = \mathbf{d}_n+h\bar{\mathbf{v}}'$$ 
$$ \delta\bar{\mathbf{u}} = \frac{h}{2}\delta\bar{\mathbf{v}} \qquad  \delta\bar{\mathbf{d}} = \frac{h}{2}\delta\bar{\mathbf{v}}'$$ 

$$\bar{\mathcal{R}} = \mathbf{R}^T\dep{\bar{\varphi}}\mathbf{C}\bar{\mathbf{e}}  \qquad \bar{\mathbf{e}} = \mathbf{R}\dep{\bar{\varphi}}\bar{\mathbf{d}}-\mathbf{e}_0 \qquad \bar{\mathcal{N}} = \mathbf{C}\bar{\mathbf{e}}  $$

$$\delta \bar{\mathcal{R}} = \frac{h}{2} (\delta\bar{\Omega}((\mathbf{R}^{(1)})^T\dep{\bar{\varphi}}\mathbf{C}\bar{\mathbf{e}} + \mathbf{R}^T\dep{\bar{\varphi}}\mathbf{CR}^{(1)}\dep{\bar{\varphi}}\bar{\mathbf{d}}) + \mathbf{R}^T\dep{\bar{\varphi}} \mathbf{CR}\dep{\bar{\varphi}}\delta\bar{\mathbf{v}}' )$$
$$\delta\bar{\mathbf{e}} = \frac{h}{2}(\delta\bar{\Omega}\mathbf{R}^{(1)}\dep{\bar{\varphi}}\bar{\mathbf{d}} + \mathbf{R}\dep{\bar{\varphi}}\delta\bar{\mathbf{v}}')$$
$$\delta\bar{\mathcal{N}} = \frac{h}{2} (\delta\bar{\Omega}\mathbf{CR}^{(1)}\dep{\bar{\varphi}}\bar{\mathbf{d}} + \mathbf{CR}\dep{\bar{\varphi}}\delta\bar{\mathbf{v}}')$$

%$$\rho_A( \mathbf{v}_{n+1}-\mathbf{v}_n) = h(\bar{\mathcal{R}}'+\bar{p}+\begin{bmatrix} 0\\ 0\\ \bar{\mathcal{N}}_1\bar\mathbf$$

V programu je napaka ker ne računam količin v $t_{n+1}$ z pravimi $u_{n+1}$.\\
{\it Najprej bi moral poračunati hitrosti za $n+1$ in $n$ in od tod hitrosti za $n+\frac{1}{2}$. Z temi izračunam $\mathbf{u}_{n+1}$ in $\mathbf{d}_{n+1}$ in od tod $\mathcal{R}_{n+1}$. Nadaljni postopek bi moral biti pravilen.}

Kaj bi moral narediti z energijama?\\
{\it Če prav razumem sta hitrosti ob časih $n+1$ in $n$ neki funkciji. Zato ker jih ne znamo izvrednostit moramo uporabiti ravnotežne enačbe $\bar{\mathcal{F}}$ in iz njih izraziti diferenco hitrosti (pospešek), kar se pojavi tudi v enačbi energije. Po drugi strani pa je funkcija vmesne hitrosti linearna kombinacija robnih.}

Kakšen bi bil pravilen postopek za račun $\delta\bar{\mathcal{R}}$, če za $\bar{\mathcal{R}}$ vzamem povprečnega?\\
{\it Ker je $\delta\mathcal{R}_n$ neodvisna od $\bar{\mathbf{v}}$ je variacija enaka $0$. Razlika med povprečnim in analitično formulo je ta, da so v vseh členih nevariirane količine tiste ob drugem oziroma vmesnem času. Najprej bi raje poiskusil doseči smiselne rezultate z izvorno formulo za $\mathcal{R}$.}





\newpage
\section{Implementacija v program}
$$\delta\bar{\mathbf{e}} = \frac{h}{2}\mathcal{P}_j
\begin{bmatrix}
0 & 0 & | \\
	0 & 0 & \mathbf{R}^{(1)}\dep{\bar{\varphi}}\bar{\mathbf{d}} \\
0 & 0 & |
\end{bmatrix} + \frac{h}{2}\mathcal{P}_j' \mathbf{R}\dep{\bar{\varphi}}$$


$$\delta\bar{\mathcal{N}} = \frac{h}{2}\mathcal{P}_j
\begin{bmatrix}
0 & 0 & | \\
0 & 0 & \mathbf{CR}^{(1)}\dep{\bar{\varphi}}\bar{\mathbf{d}} \\
0 & 0 & |
\end{bmatrix} + \frac{h}{2}\mathcal{P}_j' \mathbf{CR}\dep{\bar{\varphi}}$$

$$\delta\bar{\mathcal{R}} = \frac{h}{2}\mathcal{P}_j
\begin{bmatrix}
0 & 0 & | \\
0 & 0 & (\mathbf{R}^{(1)})^T\dep{\bar{\varphi}}\mathbf{C}\bar{\mathbf{e}} + \mathbf{R}^T\dep{\bar{\varphi}}\mathbf{CR}^{(1)}\dep{\bar{\varphi}}\bar{\mathbf{d}} \\
0 & 0 & |
\end{bmatrix} + \frac{h}{2}\mathcal{P}_j' \mathbf{R}^T\dep{\bar{\varphi}} \mathbf{CR}\dep{\bar{\varphi}}$$

Za enostaven primer konzole se komponente zadnjega stoplca in vrstice hitro povečujejo. Preveri eačbe ki prispevajo k temu delu.


\end{document}
